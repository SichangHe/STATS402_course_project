\documentclass[a4paper]{article}
\usepackage[margin=1in]{geometry}
\usepackage{graphicx}
\usepackage{xcolor}
    \newcommand{\todo}[1]{\textcolor{red}{[ #1 ]}}
    \newcommand{\instruction}[1]{\textcolor{orange}{#1}}
    % \renewcommand{\instruction}[1]{} % Uncomment to hide instructions.

\newcommand{\hidden}[1]{}

\usepackage{biblatex}
    \addbibresource{references.bib}

\usepackage[colorlinks=false]{hyperref}

\title{STATS 402 - Interdisciplinary Data Analysis\\
    Resource-Constrained Reinforcement Learning for Battlesnake\\
    Milestone Report: Stage 1
}
\author{Sichang He\\
    sichang.he@dukekunshan.edu.cn
}

\begin{document}
\maketitle

\subparagraph{Abstract}

\todo{Insert a very brief paragraph describing your project (200 words)}

\section{Project Rationale}

\todo{explain the motivation of the project
    selection by analyzing the characteristics of the selected data set,
    as well as the development trend of related fields, application prospects,
    and commonly used methods}

Battlesnake~\cite{battlesnake}
is a popular online programming competition in the form of a simultaneous
multiplayer board game. Similar to the arcade snake game,
each player controls a snake in real time on a finite grid board to be the last
one alive; the snake can change directions within each turn,
grow in length by eating randomly spawned food,
die from colliding with walls or snake bodies,
or starve to death after not eating for a long time (100 turns).
Instead of having human players control the snakes, in Battlesnake,
players develop a computer program to control their snakes' directions in each
turn, by implementing a web server that answers the game server's request.
This means players can implement their algorithms freely,
as long as they can finish answering the request within the time limitation
(500ms).

\begin{figure}
    \centering
    \includegraphics[width=0.4\linewidth]{snake_game_screenshot.png}
    \caption{A Standard Battlesnake Game with 4 Snakes on A 11x11 Board.}
    \label{fig:game}
\end{figure}

Placeholder~\cite{schulman2017proximal}

\section{Research Content and Objectives}

\todo{and critical scientific problems to be solved}

\section{Research Plan and Feasibility Analysis}

\todo{including research methods, technical routes, experimental methods,
    key technologies, etc.}

\section{Features, Innovations and the Expected Results}

\instruction{
    PENALTY FOR PLAGIARISM:\\
    \(\geq 30\%\): ZERO SCORE for the report.\\
    \((25\%,30\%\): -1.5 points\\
    \((20\%,\ 25\%\): -1.0 points\\
    (15\%, 20\%]: -0.5 points\\
    \(\leq 15\%\): no penalty, you will get 2 points
}

\printbibliography

\end{document}
