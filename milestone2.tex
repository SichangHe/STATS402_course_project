%!TEX = xelatex
\documentclass[a4paper]{article}
\usepackage[margin=1in]{geometry}
\usepackage{graphicx}
\usepackage{amsmath}
\usepackage{amssymb}
\usepackage{xcolor}
    \newcommand{\todo}[1]{\textcolor{red}{[ #1 ]}}
    \newcommand{\instruction}[1]{\textcolor{orange}{#1}}
    % \renewcommand{\todo}[1]{} % Uncomment to hide todos.
    % \renewcommand{\instruction}[1]{} % Uncomment to hide instructions.

\newcommand{\hidden}[1]{}

\usepackage{biblatex}
    \addbibresource{references.bib}

\usepackage[colorlinks=false]{hyperref}

\title{STATS 402 - Interdisciplinary Data Analysis\\
    Resource-Constrained Deep Reinforcement Learning for Battlesnake\\
    Milestone Report: Stage 2
}
\author{Steven Hé (Sīchàng)\\
    sichang.he@dukekunshan.edu.cn
}

\begin{document}
\maketitle

\instruction{
    There are no specific requirements for the stage 2 report since the progress may vary among different groups. Generally, there are four parts you need to cover in your report.
}

\subparagraph{Abstract}

Placeholder~\cite{battlesnake}.

\section{Current Status}

\todo{The current status of your project.\\
    For example, the detailed techniques you adopted to conduct the project. Has your group made any technical route adjustments? This part is essential for the groups whose actual adopted method is different from their milestone report 1. You need to explain the reason for the change.
}

\subsection{Technical Route Adjustments}

We found that Gymnasium~\cite{farama2024gymnasium}
does not support multi-agent environments like the ones we have in Battlesnake.
Therefore,
we adjusted the environment library choice to be
Pettingzoo~\cite{terry2021pettingzoo} instead.
Pettingzoo supports simultaneous-move multi-agent environments via its Parallel
API\footnote{\url{https://pettingzoo.farama.org/api/parallel/}},
and can be made compatible with Stable Baselines3~\cite{raffin2024stable}
using SuperSuit~\cite{SuperSuit}.

\subsection{Implementation Progress}

The Pettingzoo environment is implemented based on the feature extraction
described in the proposal,
using the game simulation implementation in~\cite{wrenger2024rusty}.

Notably, the features are nine 21x21 layers instead of ten,
as incorrectly stated in the proposal.

Implementing the feature extraction was challenging,
mainly because a large number of indexing is used to construct the feature
matrix layers.

\section{Data Preprocessing}

\todo{Demonstrate some initial data preprocessing results if you have.}

\section{Plan for the Next Two Weeks}

\instruction{Maximum 6 pages}

\printbibliography

\end{document}
